Симуляция системы реализована на языке программирования Python. Для создания симуляции работы протокола маршрутизации реализованы следующие абстракции:

- Router: базовый элемент сети, выполняющий основные элементы симуляции: инициализацию, отправка информации о соседях и её получение, "отключение" и прочее;

- Designated router: выделенный маршрутизатор. Управляет процессом рассылки LSA (link-state advertisement, объявление о состоянии канала) в сети. С данным маршрутизатором каждый маршрутизатор сети устанавливает отношение смежности для отправки информации об изменении сети;

- Message: структура из пары "данные"+"значение". Формируются роутерами и пересылаются по соединениям;

- Connection: пара очередей, для реализации взаимодействия.

Симуляция реализована на ЯП Python 3.10, где каждый элемент сети работает параллельно. Для ее работы устанавливается время рассылки сообщений (было принято равным 10 секунд), а также вероятность отключения узла сети (было принято равным 0.25).
