Рассмотрим пример работы программы для линейной топологии с 3 узлами. Здеь и далее: узлы – указаны их номера, связи – список номеров соседних узлов на позиции текущего узла.

\begin{itemize}
	\item Узлы [0, 1, 2]
	\item Связи [[1], [0, 2], [1]]
\end{itemize}

К сети подключены все 3 узла. Кратчайшие пути:
\begin{itemize}
	\item 0: [[0], [0, 1], [0, 1, 2]]
	\item 1: [[1, 0], [1], [1, 2]]
	\item 2: [[2, 1, 0], [2, 1], [2]]
\end{itemize}

От сети отключен 2-ой узел. Новые кратчайшие пути:
\begin{itemize}
	\item 0: [[0], [0, 1], []]
	\item 1: [[1, 0], [1], []]
	\item 2: [[], [], [2]]
\end{itemize}

Теперь рассморим пример работы программы для кольцевой топологии с 4 узлами.
\begin{itemize}
	\item Узлы [0, 1, 2, 3]
	\item Связи [[3, 1], [0, 2], [1, 3], [2, 0]]
\end{itemize}

К сети подключены все 4 узла. Кратчайшие пути:
\begin{itemize}
	\item 0: [[0], [0, 1], [0, 1, 2], [0, 3]]
	\item 1: [[1, 0], [1], [1, 2], [1, 0, 3]]
	\item 2: [[2, 1, 0], [2, 1], [2], [2, 3]]
    \item 3: [[3, 0], [3, 0, 1], [3, 2], [3]]
\end{itemize}

От сети отключен 0-ой узел. Новые кратчайшие пути из соответствующих узлов:
\begin{itemize}
	\item 1: [[], [1], [1, 2], [1, 2, 3]]
    \item 2: [[], [2, 1], [2], [2, 3]]
	\item 3: [[], [3, 2, 1], [3, 2], [3]]
\end{itemize}


Наконец, рассмотрим пример работы программы для звездной топологии с 5 узлами. Центр в узле с индексом 0.

\begin{itemize}
	\item Узлы [0, 1, 2, 3, 4]
	\item Связи [[1, 2, 3, 4], [0], [0], [0], [0]]
\end{itemize}

\begin{itemize}
	\item 0: [[0], [0, 1], [0, 2], [0, 3], [0, 4]]
	\item 1: [[1, 0], [1], [1, 0, 2], [1, 0, 3], [1, 0, 4]]
	\item 2: [[2, 0], [2, 0, 1], [2], [2, 0, 3], [2, 0, 4]]
	\item 3: [[3, 0], [3, 0, 1], [3, 0, 2], [3], [3, 0, 4]]
    \item 4: [[4, 0], [4, 0, 1], [4, 0, 2], [4, 0, 3], [4]]
\end{itemize}

От сети отключен 3-ий узел. Новые кратчайшие пути:
\begin{itemize}
	\item 0: [[0], [0, 1], [0, 2], [], [0, 4]]
	\item 1: [[1, 0], [1], [1, 0, 2], [], [1, 0, 4]]
	\item 2: [[2, 0], [2, 0, 1], [2], [], [2, 0, 4]]
	\item 4: [[4, 0], [4, 0, 1], [4, 0, 2], [], [4]]
\end{itemize}

От сети отключен 0-ой, центральный узел. Новые кратчайшие пути отсутствуют, т.к. узлы стали отдельными областями связанности:
\begin{itemize}
	\item 1: [[], [1], [], [], []]
	\item 2: [[], [], [2], [], []]
	\item 3: [[], [], [], [3], []]
    \item 4: [[], [], [], [], [4]]
\end{itemize}
