Протокол маршрутизации OSPF (Open Shortest Path First – алгоритм предпочтительного выбора кратчайшего маршрута) предназначен для работы в сетях множественного доступа, т. е. сетях, у которых может быть несколько маршрутизаторов, способных общаться друг с другом. Основой работы данного протокола является представление множества сетей, маршрутизаторов и каналов в виде ориентированного графа. Такое представление позволяет учитывать различные условия и ограничения при выборе кратчайшего пути между любыми двумя маршрутизаторами, а также делить большие системы на области, каждая из которых может обладать своей собственной топологией, условиями выбора маршрутов и другими особенностями.

Принцип работы заключается в следующем:
\begin{itemize}
	\item После включения маршрутизаторов протокол ищет непосредственно подключённых соседей и устанавливает с ними связь
	\item Затем они обмениваются друг с другом информацией о подключённых и доступных им сетях. То есть они строят карту сети (граф сети). Данная карта одинакова на всех маршрутизаторах
	\item На основе полученной информации запускается алгоритм SPF (Shortest Path First, ``выбор наилучшего пути''), который рассчитывает оптимальный маршрут к каждой сети. Данный процесс представляет из себя поиск кратчайшего пути в графе, вершинами которого являются доступные сети, а рёбрами -- пути между сетями.
\end{itemize}
