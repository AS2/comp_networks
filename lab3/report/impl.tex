Симуляция системы реализована на языке программирования Python. Исходный код доступен по следующей ссылке: \\
\hyperlink{https://github.com/AS2/comp\_networks/tree/main/lab3}{https://github.com/AS2/comp\_networks/tree/main/lab3}

Для создания симуляции работы протокола маршрутизации реализованы следующие основные абстракции:

- LossyQueue (файл lossy\_queue.py) - класс реализации очереди с потерями. Потери реализуются посредством "отказа" от добавления нового элемента с некоторой вероятностью;

- IResendBehaviour (файл resend\_protocol\_base.py) - базовый класс для реализации канала связи с повторной отправкой при потере между двумя узлами;

- GBN (файл GBN.py) - спецификация IResendBehaviour, реализующую протокол Go-Back-N;

- SP (файл SP.py) - спецификация IResendBehaviour, реализующую протокол SelectiveRepeat; 

- Connection: пара очередей, для реализации взаимодействия, вместе с реализованной политикой IResendBehaviour.

- Topology (файл topology.py) - графовое преставление сети. На основе топологии строится итоговая P2P сеть в симуляции;

- Router (файл network.py) - базовый элемент сети, выполняющий основные элементы симуляции: инициализацию, отправка информации о соседях и её получение. Информация о соседях распространяется в рамках данной симуляции отказоустойчиво;

- Designated router (файл network.py) - выделенный маршрутизатор. Его функции описаны раннее;

- FileMetadata (файл primitives.py) - структура описания файла и процесса его разбиения;

- FilePart (файл primitives.py) - структура, содержащая часть файла. Состоит из порядкового номера, размера, а также непосредственно содержания;

Симуляция реализована на ЯП Python 3.10, где каждый элемент сети работает параллельно: в течении 3 секунд запускаются узлы, после чего запускается построение кратчайших путей и постройка каналов связи, согласно топологии, а затем сборка файлов.
