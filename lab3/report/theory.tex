На сегодняшний день одним из основных методов хранения данных является централизованное хранение - хранение при помощи одного выделенного сервера. Такое хранение позволяет легко получать данные, однако имеет ряд изъянов: данные и процессы на этих компьютерах не являются прозрачными, а атака на центральный сервер может нарушить конфиденциальность.

В качестве альтернативы общественность предлагает использовать децентрализованные хранилища (такая идея входит в парадигму Web3.0). Основная идея такого метода состоит в хранении частей файлов на нескольких нодах (узлах) P2P сети. Такой подход усложняет процедуру получения файла ввиду дополнительной загрузки, однако обеспечивает защищенность: взлом одного из узлов не гарантирует утечку полного содержимого файла. В данной работе необходимо реализовать симуляцию работы децентрализованного хранилища (хранения и сборки файлов для некоторого узла сети) и продемонстрировать работу для трех различных топологий: линейной, круговой и звездочной.

В качестве симуляции работы будем представлять одноранговую сеть как некоторый неориентированный граф. Каждый элемент этого графа содержит части исходного файла. Число частей файлов обычно формируется исходя из разных желаемых факторов. 

Помимо узлов в P2P-сети с фрагментами файлов существует специальный Designated router (DR), который "дирижирует" остальными роутерами: непосредственного участия в передаче самих фрагментов файла он не принимает. В функционал данного роутера входят следующие исполняемые задачи:
\begin{itemize}
    \item Знакомство роутеров со всеми остальными роутерами сети;
    \item Координирование запросов от роутера ко всем остальным роутерам;
    \item Организация процедуры распределения частей исходного файла по узлам;
\end{itemize}
Будем считать, что каждый узел сети имеет отказоустойчивое соединение с DR, однако сами узлы могут иметь нестабильное соединение со своими соседями. Потому сами фрагменты файлов между узлами будут распространяться согласно с протоколами повторной отправки данных. В данной работе реализованы два протокола: Selective repeat (SRP) и Go-back-N (GBN) протоколы. 

Фрагменты файлов передаются от узла к узлу по кратчайшему маршруту, используя протокол Open Shortest Path First (OSPF). 

Таким образом общий процесс симуляции работы децентрализованного хранилища будет следующим:
\begin{enumerate}
    \item Инициализация топологии сети, загрузка файла, разбиение его на части и распределение частей между узлами сети.
    \item Построение карты сети для каждого роутера и кратчайщих путей посредством протокола OSPF.
    \item Построение между узлами каналов связи, обеспечивающие повторную отправку.
    \item Запуск алгоритма загрузки (сборки файла) для некоторого узла:
    \begin{enumerate}
        \item Получение необходимой информации о частях файлов с DR.
        \item Независимая параллельная передача всех частей файлов с разных роутеров сети посредством передачи по каналам связи к роутеру, на котором строится файл.
        \item Сравнение построенного файла с исходным.
    \end{enumerate}
\end{enumerate}

Промежуточные данные такие как топология сети, результаты разбиения исходного файла на части и сохранение для каждого узла, результат сборки логируются и представляются в качестве результата работы. 
