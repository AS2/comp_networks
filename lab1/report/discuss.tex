
Из графиков зависимости показателей эффективности от вероятности потерь 1-го пакета видно, что коэффициент эффективности падает быстрее у протокола Go-Back N, чем у протокола Selective Repeat. Кроме того видно, что скорость работы второго протокола выше, чем у первого.

Графики зависимостей показателей эффективности от размера окна описывают лучше различные уникальные особенности протоколов, нежели сопоставляют протоколы друг против друга: график зависимости коэффициента эффективности  для протокола Go-back N падает, в то время как тот же график для протокола Selective Repeat держится в районе константы. Такой факт несложно объясняется: Go-back N, в отличие от Selective Repeat, при отсутствии ответного сообщения ACK от потерянного пакета производит повторную отправку всех пакетов окна начиная с того пакета, который был затерян, от чего количество производимых отправок увеличивается. Selective Repeat в виду специфики построения алгоритма будет производить отправку только одного пакета - того, на отправку которого Retriever не вернкул ACK сообщение. 

Второй график из этой пары показывает другую закономерность: время работы протокола Selective Repeat уменьшается по мере увеличения окна, в то время как время работы Go-back N имеет колеблется, но в районе постоянного значения. Это тоже несложно объяснить: по мере увеличения окна протокол Selective Repeat может отправлять больше пакетов "единовременно" (или, как будет правильнее сказать, во время ожидания истекания lost\_timeout некоторого потерянного пакета). Для протокола Go-back N закономерность можно объясняется работой протокола: при потере хотя бы одного пакета из окна, все пакеты, начиная с этого, отправляются повторно. Увеличение окна конечно и влияет на количество отправок секторов, ровно как и на количество потерянных пакетов в совокупности, но не влияет на вероятность потери пакета, с которого начинается окно: данная вероятность у нас постоянна, а потери прочих пакетов, находящихся в окне после самого первого потерянного, не влияет на время ожидания, поскольку таймер для первого потерянного в окне пакета истечет раньше, чем последующего.