Протоколы и эмуляторы исполнителей реализованы в двух отдельных потоках на языке Python. Обмен данными осуществляется через очередь сообщений. Программа состоит из следующих элементов:

\begin{itemize}
	\item Message -- "сообщение", эквивалент пакета с номинальным, минимальным набором информации. Хранит в себе статус при передаче ("ok" - передан успешно, "lost" - потерян), а также такую дополнительную информацию как свой порядковой номер в окне, уникальный номер блока; 
    \item LossyMessageQueue -- канал коммуникации, который может с некоторой заданной вероятностью от 0 до 1 "терять" переданный в эту очередь пакет (где 0 - нет потерь, а 1 - потери всегда); 
	\item Sender -- отправитель, формирует сообщения с данными;
	\item Reciever -- получатель, получает сообщения и сообщает о факте
	доставки.
\end{itemize}

Sender и Reciever представлены в двух вариантах, которые реализуют протоколы Go-back N и Selective repeat сответственно. Система в рамках лабораторной работы принимала следующие параметры:
\begin{itemize}
	\item protocol -- протокол связи (Go-back N или Selective repeat);
	\item window\_size -- величина окна в выбранном протоколе;
	\item lost\_timeout -- время в секундах, после которого пакет считается утерянным в случае отсутствия подтверждения его доставки;
	\item loss\_probability -- вероятность потери сообщения при передаче [0, 1].
\end{itemize}