\documentclass[14pt,a4paper,article]{ncc}
\usepackage[a4paper, mag=1000, left=2.5cm, right=1cm, top=2cm, bottom=2cm, headsep=0.7cm, footskip=1cm]{geometry}
\usepackage[utf8]{inputenc}
\usepackage[T2A]{fontenc}
\usepackage[english,russian]{babel}
\usepackage{indentfirst}
%\usepackage[dvipsnames]{xcolor}
\usepackage{amsfonts} 
\usepackage{amssymb} 
\usepackage{amsmath, etoolbox}
\usepackage{graphicx}
\usepackage{float}
\graphicspath{{../figure/}}
\DeclareGraphicsExtensions{.png,.jpg, .jpeg}

%\bibliographystyle{gost-numeric.bbx}
\usepackage{csquotes}
\usepackage[backend=biber]{biblatex}
\addbibresource{literature.bib}

\usepackage{fancyhdr}
\pagestyle{fancy}
\fancyhead[LE,RO]{\thepage}
\fancyfoot{} 

\usepackage{listings}

%\patchcmd\subequations
%{\theparentequation\alph{equation}}
%{\subequationsformat}
%{}{}

%\newcommand{\subequationsformat}{\theparentequation.\arabic{equation}}

%\numberwithin{equation}{subsection}


\usepackage[colorlinks]{hyperref}
\hypersetup{linkcolor=black}

\begin{document}

% Title page 
\begin{titlepage}
    \begin{center}
        \textsc{
            Санкт-Петербургский политехнический университет Петра Великого \\[5mm]
            Физико-механический институт\\[2mm]
            Высшая школа прикладной математики и вычислительной физики
        }   
        \vfill
        \textbf{\large
            Компьютерные сети\\
            Отчёт по лабораторной работе №1 \\
            ``Реализация протоколов автоматического запроса повторной передачи
            Go-Back-N и Selective Repeat'' \\[3mm]
            %по курсовой работе \\[3mm]
        }                
    \end{center}

    \vfill
    \hfill
    \begin{minipage}{0.5\textwidth}
        Выполнил: \\[2mm]   
		Студент: Сачук Александр Сергеевич \\
		Группа: 5040102/30201\\
    \end{minipage}

	\hfill
	\begin{minipage}{0.5\textwidth}
		Принял: \\[2mm]
		к. ф.-м. н., доцент \\   
		Баженов Александр Николаевич
	\end{minipage}

    \vfill
    \begin{center}
        \theyear\ г.
    \end{center}
\end{titlepage}

\tableofcontents
\listoffigures
%\listoftables
\newpage

\section{Постановка задачи}
Требуется реализовать протокол маршрутизации OSPF и проверить работоспособность протокола для следующих видов топологии: линейная, кольцевая, звёздная. Проверить возможность перестройки таблиц достижимости в случае стохастического разрыва связи.

\section{Теория}
Протокол маршрутизации OSPF (Open Shortest Path First – алгоритм предпочтительного выбора кратчайшего маршрута) предназначен для работы в сетях множественного доступа, т. е. сетях, у которых может быть несколько маршрутизаторов, способных общаться друг с другом. Основой работы данного протокола является представление множества сетей, маршрутизаторов и каналов в виде ориентированного графа. Такое представление позволяет учитывать различные условия и ограничения при выборе кратчайшего пути между любыми двумя маршрутизаторами, а также делить большие системы на области, каждая из которых может обладать своей собственной топологией, условиями выбора маршрутов и другими особенностями.

Принцип работы заключается в следующем:
\begin{itemize}
	\item После включения маршрутизаторов протокол ищет непосредственно подключённых соседей и устанавливает с ними связь
	\item Затем они обмениваются друг с другом информацией о подключённых и доступных им сетях. То есть они строят карту сети (граф сети). Данная карта одинакова на всех маршрутизаторах
	\item На основе полученной информации запускается алгоритм SPF (Shortest Path First, ``выбор наилучшего пути''), который рассчитывает оптимальный маршрут к каждой сети. Данный процесс представляет из себя поиск кратчайшего пути в графе, вершинами которого являются доступные сети, а рёбрами -- пути между сетями.
\end{itemize}


\section{Реализация}
Симуляция системы реализована на языке программирования Python. Исходный код доступен по следующей ссылке: \\
\hyperlink{https://github.com/AS2/comp\_networks/tree/main/lab3}{https://github.com/AS2/comp\_networks/tree/main/lab3}

Для создания симуляции работы протокола маршрутизации реализованы следующие основные абстракции:

- LossyQueue (файл lossy\_queue.py) - класс реализации очереди с потерями. Потери реализуются посредством "отказа" от добавления нового элемента с некоторой вероятностью;

- IResendBehaviour (файл resend\_protocol\_base.py) - базовый класс для реализации канала связи с повторной отправкой при потере между двумя узлами;

- GBN (файл GBN.py) - спецификация IResendBehaviour, реализующую протокол Go-Back-N;

- SP (файл SP.py) - спецификация IResendBehaviour, реализующую протокол SelectiveRepeat; 

- Connection: пара очередей, для реализации взаимодействия, вместе с реализованной политикой IResendBehaviour.

- Topology (файл topology.py) - графовое преставление сети. На основе топологии строится итоговая P2P сеть в симуляции;

- Router (файл network.py) - базовый элемент сети, выполняющий основные элементы симуляции: инициализацию, отправка информации о соседях и её получение. Информация о соседях распространяется в рамках данной симуляции отказоустойчиво;

- Designated router (файл network.py) - выделенный маршрутизатор. Его функции описаны раннее;

- FileMetadata (файл primitives.py) - структура описания файла и процесса его разбиения;

- FilePart (файл primitives.py) - структура, содержащая часть файла. Состоит из порядкового номера, размера, а также непосредственно содержания;

Симуляция реализована на ЯП Python 3.10, где каждый элемент сети работает параллельно: в течении 3 секунд запускаются узлы, после чего запускается построение кратчайших путей и постройка каналов связи, согласно топологии, а затем сборка файлов.


\section{Результаты}
Рассмотрим пример работы программы для линейной топологии с 3 узлами. Здеь и далее: узлы – указаны их номера, связи – список номеров соседних узлов на позиции текущего узла.

\begin{itemize}
	\item Узлы [0, 1, 2]
	\item Связи [[1], [0, 2], [1]]
\end{itemize}

К сети подключены все 3 узла. Кратчайшие пути:
\begin{itemize}
	\item 0: [[0], [0, 1], [0, 1, 2]]
	\item 1: [[1, 0], [1], [1, 2]]
	\item 2: [[2, 1, 0], [2, 1], [2]]
\end{itemize}

От сети отключен 2-ой узел. Новые кратчайшие пути:
\begin{itemize}
	\item 0: [[0], [0, 1], []]
	\item 1: [[1, 0], [1], []]
	\item 2: [[], [], [2]]
\end{itemize}

Теперь рассморим пример работы программы для кольцевой топологии с 4 узлами.
\begin{itemize}
	\item Узлы [0, 1, 2, 3]
	\item Связи [[3, 1], [0, 2], [1, 3], [2, 0]]
\end{itemize}

К сети подключены все 4 узла. Кратчайшие пути:
\begin{itemize}
	\item 0: [[0], [0, 1], [0, 1, 2], [0, 3]]
	\item 1: [[1, 0], [1], [1, 2], [1, 0, 3]]
	\item 2: [[2, 1, 0], [2, 1], [2], [2, 3]]
    \item 3: [[3, 0], [3, 0, 1], [3, 2], [3]]
\end{itemize}

От сети отключен 0-ой узел. Новые кратчайшие пути из соответствующих узлов:
\begin{itemize}
	\item 1: [[], [1], [1, 2], [1, 2, 3]]
    \item 2: [[], [2, 1], [2], [2, 3]]
	\item 3: [[], [3, 2, 1], [3, 2], [3]]
\end{itemize}


Наконец, рассмотрим пример работы программы для звездной топологии с 5 узлами. Центр в узле с индексом 0.

\begin{itemize}
	\item Узлы [0, 1, 2, 3, 4]
	\item Связи [[1, 2, 3, 4], [0], [0], [0], [0]]
\end{itemize}

\begin{itemize}
	\item 0: [[0], [0, 1], [0, 2], [0, 3], [0, 4]]
	\item 1: [[1, 0], [1], [1, 0, 2], [1, 0, 3], [1, 0, 4]]
	\item 2: [[2, 0], [2, 0, 1], [2], [2, 0, 3], [2, 0, 4]]
	\item 3: [[3, 0], [3, 0, 1], [3, 0, 2], [3], [3, 0, 4]]
    \item 4: [[4, 0], [4, 0, 1], [4, 0, 2], [4, 0, 3], [4]]
\end{itemize}

От сети отключен 3-ий узел. Новые кратчайшие пути:
\begin{itemize}
	\item 0: [[0], [0, 1], [0, 2], [], [0, 4]]
	\item 1: [[1, 0], [1], [1, 0, 2], [], [1, 0, 4]]
	\item 2: [[2, 0], [2, 0, 1], [2], [], [2, 0, 4]]
	\item 4: [[4, 0], [4, 0, 1], [4, 0, 2], [], [4]]
\end{itemize}

От сети отключен 0-ой, центральный узел. Новые кратчайшие пути отсутствуют, т.к. узлы стали отдельными областями связанности:
\begin{itemize}
	\item 1: [[], [1], [], [], []]
	\item 2: [[], [], [2], [], []]
	\item 3: [[], [], [], [3], []]
    \item 4: [[], [], [], [], [4]]
\end{itemize}


\section{Обсуждение}

Из графиков зависимости показателей эффективности от вероятности потерь 1-го пакета видно, что коэффициент эффективности падает быстрее у протокола Go-Back N, чем у протокола Selective Repeat. Кроме того видно, что скорость работы второго протокола выше, чем у первого.

Графики зависимостей показателей эффективности от размера окна описывают лучше различные уникальные особенности протоколов, нежели сопоставляют протоколы друг против друга: график зависимости коэффициента эффективности  для протокола Go-back N падает, в то время как тот же график для протокола Selective Repeat держится в районе константы. Такой факт несложно объясняется: Go-back N, в отличие от Selective Repeat, при отсутствии ответного сообщения ACK от потерянного пакета производит повторную отправку всех пакетов окна начиная с того пакета, который был затерян, от чего количество производимых отправок увеличивается. Selective Repeat в виду специфики построения алгоритма будет производить отправку только одного пакета - того, на отправку которого Retriever не вернкул ACK сообщение. 

Второй график из этой пары показывает другую закономерность: время работы протокола Selective Repeat уменьшается по мере увеличения окна, в то время как время работы Go-back N имеет колеблется, но в районе постоянного значения. Это тоже несложно объяснить: по мере увеличения окна протокол Selective Repeat может отправлять больше пакетов "единовременно" (или, как будет правильнее сказать, во время ожидания истекания lost\_timeout некоторого потерянного пакета). Для протокола Go-back N закономерность можно объясняется работой протокола: при потере хотя бы одного пакета из окна, все пакеты, начиная с этого, отправляются повторно. Увеличение окна конечно и влияет на количество отправок секторов, ровно как и на количество потерянных пакетов в совокупности, но не влияет на вероятность потери пакета, с которого начинается окно: данная вероятность у нас постоянна, а потери прочих пакетов, находящихся в окне после самого первого потерянного, не влияет на время ожидания, поскольку таймер для первого потерянного в окне пакета истечет раньше, чем последующего.

\printbibliography
%\addcontentsline{toc}{section}{Литература}

\section{Приложения} \label{app}

\begin{enumerate}
	\item Репозиторий с кодом программы и кодом отчёта:
	
	\href{https://github.com/AS2/comp_networks/tree/develop/lab2/}{https://github.com/AS2/comp_networks/tree/develop/lab2/}

\end{enumerate}


\end{document}
